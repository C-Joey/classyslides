\documentclass{beamer} 

  \usepackage{filecontents}
  
  % Select the classyslides base, fonts and dark theme.
  \usetheme{classyslidesbase}
  \usetheme{classyslidesfonts}
  \mode<presentation>
  
  \title{There Is No Largest Prime Number}
  \author[Euclid]{Euclid of Alexandria \\ euclid@alexandria.edu}
  \date{\today}
  
  % Bibliography
  \begin{filecontents}{euclid.bib}
    @book{Labov1972,
      Address = {Philadelphia},
      Author = {William Labov},
      Publisher = {University of Pennsylvania Press},
      Title = {Sociolinguistic Patterns},
      Year = {1972}}
  
  @book{Chomsky1957,
      Address = {The Hague},
      Author = {Noam Chomsky},
      Publisher = {Mouton},
      Title = {Syntactic Structures},
      Year = {1957}}
  }
  \end{filecontents}
  \addbibresource{./euclid.bib}
  
  %!%!%!%!%!%!%
  % document  %
  %!%!%!%!%!%!%
  
  \begin{document}
  
  \begin{frame}
  \titlepage
  \end{frame}
  
  %! Table of contents.
  
  \contentpagetrue
  \begin{frame}
  \frametitle{Outline}
  \tableofcontents
  \end{frame}
  
  \section{Motivation}
  \subsection{The Basic Problem That We Studied}
  \contentpagefalse
  \frame{\sectionpage}
  
  \contentpagetrue
  \begin{frame}
    \frametitle{What Are Prime Numbers?}
    \begin{Definition}{Prime number}
      A \emph{prime number} is a number that has exactly two divisors.
    \end{Definition}
    \begin{Example}
      \begin{itemize}
        \item 2 is prime (two divisors: 1 and 2).
        \item 3 is prime (two divisors: 1 and 3).
        \item 4 is not prime (\alert{three} divisors: 1, 2, and 4).
      \end{itemize}
    \end{Example}
  \end{frame}
  
  \begin{frame}[fragile]
    \frametitle{There Is No Largest Prime Number}%
    \begin{Theorem}{Prime numbers}
      There is no largest prime number.
    \end{Theorem}
  \end{frame}
  
  
  \begin{frame}[fragile]
    \frametitle{There Is No Largest Prime Number}%
    \begin{Proof}
      \begin{enumerate}
        \item<1-> Suppose $p$ were the largest prime number.
        \item<2-> Let $q$ be the product of the first $p$ numbers.
        \item<3-> Then $q + 1$ is not divisible by any of them.
        \item<4-> But $q + 1$ is greater than $1$, thus divisible by some prime number not in the first $p$ numbers.\qedhere
      \end{enumerate}
    \end{Proof}
    \uncover<4->{The proof used \textit{reductio ad absurdum}.}
  \end{frame}
  
  \begin{frame}[t]
    \frametitle{What's Still To Do?}
    \begin{itemize}
      \item Answered Questions
      \begin{itemize}
        \item How many primes are there?
      \end{itemize}
      \item Open Questions
      \begin{itemize}
        \item Is every even number the sum of two primes?
      \end{itemize}
    \end{itemize}
  \end{frame}
  
  \begin{frame}[fragile]
    \frametitle{An Algorithm For Finding Prime Numbers.}
    \begin{Code}[C++]{\texttt{FindPrimeNumbers}}
  int main (void)
  {
    std::vector<bool> is_prime (100, true);
    for (int i = 2; i < 100; i++)
    if (is_prime[i])
    {
      std::cout << i << " ";
      for (int j = i; j < 100; is_prime [j] = false, j+=i);
      }
      return 0;
  }
    \end{Code}
    \begin{uncoverenv}<2>
      Note the use of \verb|std::|.
    \end{uncoverenv}
  \end{frame}
  
  \nocite{*}
  \begin{frame}[allowframebreaks]
    \frametitle{References}
    \printbibliography
  \end{frame}
  \end{document}